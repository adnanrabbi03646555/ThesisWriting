\chapter{Conclusion and Future Work}
A keyword-based annotation language that can be used out of the box for annotating UML state charts and C code in two software development phases by providing two editors for inserting security annotations in order to detect information flow bugs automatically. It's evaluated on some sample programs and showed that this approach is applicable to real life scenarios.\\
It's a light-weight annotation language usable for specifying
information flow security constraints which can be used in the
design and coding phase in order to detect information flow
bugs.\\
In future it can be extended for source code editor as
a pop-up window based proposal editor used to add/retrieve
annotation to/from a library. The definition of new language
annotation tags should be possible from the same window by
providing two running modes (language extension mode and
annotation mode). The envisaged result is to reduce the gap
between annotations insertion/retrieval and the definition of
new language tags. This would help to create personalized
annotated libraries which can be collaboratively annotated if
needed.