\chapter{Background Information}
\section{Sanitization}
Sanitization is the process of removing sensitive information from a document or other message or sometimes encrypting messages, so that the document may be distributed to a broader audience.Sometimes sanitization can be called as an operation that ensures that user input can be safely used in an SQL query. Some basic purpose of sanitization are given below:
\begin{itemize}
	\item To identify the set of parameters and global variables which must be sanitized before calling functions.
	\item It is acceptable to first pass the untrusted user input through a trusted sanitization function.	
	\item Any user input data must flow through a sanitization function before it flows into a SQL query.
	\item Confidential data needs to be cleansed to avoid information leaks.
	\item Most paths that go from a source to a sink pass through a sanitizer.
	\item Developers typically define a small number of sanitization functions or use ones supplied in libraries.
\end{itemize}

\section{Declassification}
Information security has a challenge to address: enabling information flow controls with expressive information release (or declassification) policies. In a scenario of systems that operate on data with different sensitivity levels, the goal is to provide security assurance via restricting the information flow within the system.\\
To declassify information means lowering the security classification of selected information. Sabelfeld and Sands \cite{ref_1_sabelfeld2009declassification} identify four different dimensions of declassification, what is declassified, who is able to declassify, where the declassification occurs and when the declassification takes place.

