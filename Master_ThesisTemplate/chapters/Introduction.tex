\chapter{Introduction}
\label{chapter:Introduction}

The detection of information flow vulnerabilities uses dynamic analysis techniques , static analysis techniques and hybrid techniques which combine static and dynamic approaches. The static techniques need to know when to use  sanitization , declassification and authentication functions.
A solution for tagging sanitization, declassification and authentication in source code is based on libraries which contain all needed annotations attached to function declarations. This approach plays an important role mainly for static analysis bug detection techniques where the information available during program run-time is not available nor the interaction with the environment can be fully simulated.
Extended Static Checking (ESC) is a promising research area which tries to cope with the shortage of not having the program run-time information. During extended static analysis additional information is provided to the static analysis process. This information can be used to define trust boundaries and tag variables. Textual annotations are usually manually added by the user in source code. At the same time annotations can be automatically generated and inserted into source code . ESC can be used to eliminate bugs in a late stage of the software project when code development is finished. Tagging and checking for information exposure bugs during the design phase would eliminate the potential of implementing software bugs which can only be removed very costly after wards. Thus security concerns should be enforced into source code right after the conceptual phase of the project.
The paper presents five challenges concerning ESC. The last challenge reports of the annotation as being a very time consuming burden and is therefore disliked by some programming
teams. The authors argue about the fact that annotations can cover design decisions and enhance the quality of source code. We argue that annotations are necessary in order to do ESC and the user needs a kind of assistance tool that helps selecting the suited annotation based on the current context. Thus the annotation burden needed for learning and applying the language should be reduced. At the same time adding annotations to reusable code libraries reduces even more the annotation burden since libraries can be reused, shared and changed by software development
teams.
 
 \cite{haykin2004comprehensive}
\section{Latex Introduction}
There is no need for a latex introduction since there is plenty of literature out there.
 


