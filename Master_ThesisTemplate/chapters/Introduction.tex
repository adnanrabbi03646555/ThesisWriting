\chapter{Introduction}
\label{chapter:Introduction}
Security is one of the important factor in software development. To develop a secure system is not an easy task. Only adding some information flow restrictions is not sufficient. Information flow vulnerabilities detection in code and UML state charts is not well known. It is particularly one of the challenging issue now-a-days. Actually there is no common annotation language for annotating UML state charts and source code with
information flow security constraints such that errors can be detected also when code is not available. Also there are
no automated checking tools which can reuse the annotated constraints in early stages of software development phase to check
for information flow errors. It is important to specify security constraints as early as possible in the software
development phase in order to avoid later costly repairs or exploitable vulnerabilities.\\

A solution for tagging sanitization, declassification and authentication in source code is based on libraries which contain all needed annotations attached to function declarations. This approach plays an important role mainly for static analysis bug detection techniques where the information available during program run-time. Detection of information flow vulnerabilities uses dynamic analysis techniques , static analysis techniques and hybrid techniques which combine static and dynamic approaches. The static techniques need to know when to use  sanitization , declassification and authentication functions. Data sanitization has been studied in the context of architectures for high assurance systems, language-based information flow controls and privacy-preserving data publication \cite{ref_1_gehani2011f}. A global policy of noninterference which ensures that high-security data will not be observable on low-security channels. Because noninterference is typically
too strong a property, most programs use some form of declassification to selectively leak high security  information \cite{ref_2_hicks2006trusted}. Declassification is often expressed as an operation within a given
program. Authentication is the way through which the users get access to a system. In this research main focus are these three types of functionalities which are sanitization, declassification and authentication errors in UML state charts. \\

Web applications are often implemented by developers with limited security skills and that's why they contain vulnerabilities. Most of these vulnerabilities come from the lack of input validation. That is, web applications use malicious input as part of a sensitive operation, without having properly checked or sanitized the input values
prior to their use. Another function is declassification. We all know that computing systems often deliberately release (or declassify) sensitive information. A main security concern for systems permitting information release is whether this release is safe or not. Is it possible that the attacker compromises the information release mechanism and extracts more secret information than intended? Now-a-days computing systems release sensitive information by classifying the basic goals according to what information is released, who releases information, where in the system information is released and when information can be released. in case of authentication, it is the mechanism actually which confirms the identity of users trying to access a system( application, login verification into a system, database access etc.). \\

It is important to develop techniques and tools which can detect information flow type of errors before software developers or programmers develop their production code. Information flow errors in UML models and code are introduced by software developers or programmers who are sometimes unaware or blind while developing software. This type of vulnerabilities are hard to detect because static code analysis techniques need previous knowledge about what should be considered a security issue. Code annotations which are added mainly during software development \cite{ref_18_chess2004static} can be used to provide additional knowledge regarding security issues. On the other hand code annotations can increase the number of source code lines by 10\%. In order to detect information flow vulnerabilities software artifacts have to be annotated with annotations attached to public data, private data and to system trust boundaries. Next, annotated artifacts have to be made tractable by tools which can use the annotations and check if information flow constraints hold or not based on information propagation techniques.\\

Static Checking is a promising research area which tries to cope with the shortage of not having the program run-time information. During extended static analysis additional information is provided to the static analysis process. This information can be used to define trust boundaries and tag variables. Textual annotations are usually manually added by the user in source code. At the same time annotations can be automatically generated and inserted into source code . Static Checking can be used to eliminate bugs in a late stage of the software project when code development is finished. Tagging and checking for information exposure bugs during the design phase would eliminate the potential of implementing software bugs which can only be removed very costly after wards. Thus security concerns should be enforced into source code right after the conceptual phase of the project. \\

It can be said that annotations can cover design decisions and enhance the quality of source code. Annotations are necessary in order to do Static Checking and the user needs a kind of assistance tool that helps selecting the suited annotation based on the current context.At the same time adding annotations to reusable code libraries reduces even more the annotation burden since libraries can be reused, shared and changed by software development teams.\\

In summary the contribution for this research are:
\begin{itemize}
 \item A novel light-weight security annotation language
	used to define information flow constraints regarding authentication, declassification and santization function errors  in UML state charts and source code.
	
\item Annotation language editors designed as Eclipse
	plug-ins which is used to edit UML state charts and
	source code files.
	
\item Source code generator developed as Eclipse plug-in which is used to generate C code with header files from UML State chart.
	
\item Experiments are presented in experiments section based on
	automatic loading and usage of textual annotations
	inside 3 new checkers.
	
\end{itemize}

 


