\chapter{Limitations}
The main limitations of this system are given below:
\begin{itemize}
	\item \textbf{Function calls and statements are possible to model in state chart editor:} Now in UML state chart user can model function calls and statements like variable declaration of C/C++ language. But it is not possible to model switch-case statements, loops etc. In future it may be included.
	
	\item \textbf{Region names are fixed:} To model the source code user have to include some regions as per the characteristics of state chart. Inside the region user can add states, transition, composite states, final state, initial state etc. But according to this research now user can declare two regions. Currently user have to model a real life system using region name good\_path() and bad\_path() or one of them in UML state chart editor. Otherwise code generator will not work. With these two region user can model the source code of a real system as a UML state chart. 
	
	\item \textbf{Generator generates two files:} After modeling the source code as UML state charts need to generate the code from the model. The generator was built with Eclipse xtend. Generator included inside YAKINDU SCT Editor.Now code generator generates only two files. One is .c file and another one is .h file.
	
	\item \textbf{Generator included inside YAKINDU SCT editor:} As an open source tool here for this research we chose YAKINDU SCT Editor to model the source code into state charts. Inside the YAKINDU SCT Editor code generator also exist for C and C++ language. We developed the source code generator according to the requirements inside Myc package for C code generation. Now code generator works with YAKINDU SCT editor.
	
	
	\item \textbf{Bindings with operating system:}  Code generator, UML statechart modeling developed in windows operating system. Both of these are now working only in windows operating system.
	
	\item \textbf{Simulation not working:} The simulation for statemachine is working for normal state chart. But the simulation of UML statechart is not currently working in selected scenarios in state chart editor.
	
	\item \textbf{Fixed function names:} For this research we had to add some more type of functions like authentication, declassification and santization to annotate the UML state chart as well as source code. These three types of function has included inside the annotation language grammar. Now source, sink, authentication, declassification and santization types of function can be annotated. So, user's can now be able to annotate only these type of functions not more than that.
	
\end{itemize}